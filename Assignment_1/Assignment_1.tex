% Options for packages loaded elsewhere
\PassOptionsToPackage{unicode}{hyperref}
\PassOptionsToPackage{hyphens}{url}
%
\documentclass[
]{article}
\usepackage{amsmath,amssymb}
\usepackage{iftex}
\ifPDFTeX
  \usepackage[T1]{fontenc}
  \usepackage[utf8]{inputenc}
  \usepackage{textcomp} % provide euro and other symbols
\else % if luatex or xetex
  \usepackage{unicode-math} % this also loads fontspec
  \defaultfontfeatures{Scale=MatchLowercase}
  \defaultfontfeatures[\rmfamily]{Ligatures=TeX,Scale=1}
\fi
\usepackage{lmodern}
\ifPDFTeX\else
  % xetex/luatex font selection
\fi
% Use upquote if available, for straight quotes in verbatim environments
\IfFileExists{upquote.sty}{\usepackage{upquote}}{}
\IfFileExists{microtype.sty}{% use microtype if available
  \usepackage[]{microtype}
  \UseMicrotypeSet[protrusion]{basicmath} % disable protrusion for tt fonts
}{}
\makeatletter
\@ifundefined{KOMAClassName}{% if non-KOMA class
  \IfFileExists{parskip.sty}{%
    \usepackage{parskip}
  }{% else
    \setlength{\parindent}{0pt}
    \setlength{\parskip}{6pt plus 2pt minus 1pt}}
}{% if KOMA class
  \KOMAoptions{parskip=half}}
\makeatother
\usepackage{xcolor}
\usepackage[margin=1in]{geometry}
\usepackage{graphicx}
\makeatletter
\def\maxwidth{\ifdim\Gin@nat@width>\linewidth\linewidth\else\Gin@nat@width\fi}
\def\maxheight{\ifdim\Gin@nat@height>\textheight\textheight\else\Gin@nat@height\fi}
\makeatother
% Scale images if necessary, so that they will not overflow the page
% margins by default, and it is still possible to overwrite the defaults
% using explicit options in \includegraphics[width, height, ...]{}
\setkeys{Gin}{width=\maxwidth,height=\maxheight,keepaspectratio}
% Set default figure placement to htbp
\makeatletter
\def\fps@figure{htbp}
\makeatother
\setlength{\emergencystretch}{3em} % prevent overfull lines
\providecommand{\tightlist}{%
  \setlength{\itemsep}{0pt}\setlength{\parskip}{0pt}}
\setcounter{secnumdepth}{-\maxdimen} % remove section numbering
\ifLuaTeX
  \usepackage{selnolig}  % disable illegal ligatures
\fi
\usepackage{bookmark}
\IfFileExists{xurl.sty}{\usepackage{xurl}}{} % add URL line breaks if available
\urlstyle{same}
\hypersetup{
  pdftitle={Assignment\_1},
  hidelinks,
  pdfcreator={LaTeX via pandoc}}

\title{Assignment\_1}
\author{}
\date{\vspace{-2.5em}2024-03-12}

\begin{document}
\maketitle

\section{Milestone 1: Analysis rental hosing on market in New
Zealand/Aotearoa}\label{milestone-1-analysis-rental-hosing-on-market-in-new-zealandaotearoa}

\subsection{Objective}\label{objective}

\subsubsection{The ojective of my project is to analysis the rental
market trends in New zealand/Aotearoa in pass 30 years(1993 -
2023)}\label{the-ojective-of-my-project-is-to-analysis-the-rental-market-trends-in-new-zealandaotearoa-in-pass-30-years1993---2023}

\subsection{Data}\label{data}

My data analysis project will focused on Aotearoa rental market, I will
utilize data obtained from mainly
\href{https://www.tenancy.govt.nz/about-tenancy-services/data-and-statistics/rental-bond-data/}{tenancy
servicer} collection and privoing open resource dataset. In the given
dataset it records 66 regoin with time ,median rentail price, mean
rentail price, and differenct kinds of bonds.

By spcicify different kind of bonds are able to identitfy the rent hours
move in status, lodged\_bonds can be idenitify as aecurity deposites
held by landloards or managing agents to safeguard the potential damages
or unpaied rent during a tenancy. active\_bond, refer Security deposite
currently in use to secure rental properties which means the house is
renting now. Cloes bond security deposits returned or no longer in
effect following the end of a tenancy.

In addition in the datset also given upper quartile rent which refers to
the 75th percentile of rental price; lower quartile rent refers to the
25th percentile of rental price; and logarithm of standard devation of
weekly rent which refers to the logarithms of the standard devation of
weekly rental price,

\subsection{Exploratory Ideas}\label{exploratory-ideas}

\begin{enumerate}
\def\labelenumi{\arabic{enumi}.}
\tightlist
\item
  Trend analysis of rental prices across regions over time:

  \begin{itemize}
  \tightlist
  \item
    explore how median and mean tenal price changes over time across in
    Aotearoa
  \item
    compare the distribution of rental prices in different regions
  \end{itemize}
\item
  Analysis of rental price variability and affordability

  \begin{itemize}
  \tightlist
  \item
    calculate the upper and lower quarile rent for each regoin and
    examine their distrbution
  \item
    Identify regions with high rental price variability or affordability
    challenges and poential facotors contributing to these trends.
  \end{itemize}
\end{enumerate}

\subsection{Approach}\label{approach}

To start with, I need to have enough data from NZ's rental market. After
wrangling and tidying data set, I am able to extract potental useful
dataset into a human friendly format. Next setp is to filter unneed data
to make my spread sheet from human friendly format into machine friendly
format which requires all the columns are atomic.

\subsection{Challenges}\label{challenges}

\begin{itemize}
\tightlist
\item
  Use the NZ rental bond data to investigate if the law change in Feb
  did lead to more rentals drop off the market resulting in less
  rentals.
\end{itemize}

\end{document}
